\documentclass[12pt]{article}
\usepackage{fullpage}
\usepackage{hyperref}
\usepackage{url}

\title{Tips for TestDaF}

\begin{document}

\setlength{\parindent}{0pt}

\maketitle

\def\myspace{\vspace{0.7em}}

TestDaF is a german language test, that is a prerequisite for university admissions. 
Therefore, the main topics it covers have to do with academic and social life at a german university.\\

Here are a few examples of big topics covered in the exam:
\begin{itemize}
    \item student life (studentisches Leben)
        \begin{itemize}
            \item financing the studies (Studiumfinanzierung), 
            \item getting an apartment (Wohnungssuche)
            \item getting to know other students (das Kennenlernen)
            \item family, gender (Familie, Gender)         
        \end{itemize}
        
    \item university (Universit\"at)
        \begin{itemize}
            \item admissions office (Zulassungsb\"uro)
            \item application (Bewerbung, Zulassung und Einschreibung)
            \item student services (Studienberatung)
            \item studies structure: seminars, lectures, exams (Seminare, Vorlesungen, P\"ufungen/Klausuren)     
            \item faculty and professors (Fakult\"aten und Professoren)       
        \end{itemize}

    \item environment (Umwelt)
        \begin{itemize}
            \item climate (Klima), climate change (Klimawandel)
            \item nature and nature protection (Natur und Naturschutz)
            \item birds (V\"ogel)
            \item animals (Tiere)
            \item globalisation (Globalisierung)      
        \end{itemize}
        
    \item science (Wissenschaft)
        \begin{itemize}
            \item medicine (Medizin)
            \item biology (Biologie)
            \item antropology (Antropologie)
            \item physics (Physik)
            \item social sciences (soziale Wissenschaften)
            \item psychology (Psychologie)        
        \end{itemize}
    \item health and sports (Gesundheit und Sport)
\end{itemize}

Each of these topics have a specific set of vocabulary (Wortschatz) that is important to know in order to understand the texts
and to be able to express oneself in the essay.

In table~\ref{table:uni} you can find a short vocabulary list on the topic 'university'. 
It is important, however, to read the example texts and learn more words.

\begin{table}
\begin{tabular}{l | l}
    der/die Absolvent/-in   & alumnus \\
    der/die Abiturient/-in  & [no equivalent in english] means a person who has Abitur \\
    die Bewerbung           & application\\
    die Zulassung           & admissions\\
    das Zulassungsb\"uro    & admissions office\\
    die Zulassungskriterien & admission requirements\\
    die Einschreibung       & enrollment\\
    der Studiengang         & major\\
    das Studienfach         & major\\
    die Fachwechsel         & major change\\
    die Bewerbungsfrist     & application deadline\\
    der Zugang zum Studium  & admission to studies\\
    der Semesterticket      & semester ticket\\
    die Erm\"a{\ss}igung    & discount\\
    der Nebenjob            & job on the studies\\
    der Minijob             & job for max. 450 euros\\
    das BAf\"og             & very important german scholarship system\\
    der Studienkredit       & student loan\\
    die F\"orderung         & support\\
    die Studienberatung     & student servives\\
    das Pr\"ufungsamt       & examination office\\
    die Pr\"ufung           & exam\\
    die Klausur             & only a written exam\\
    der Studienabschluss    & graduation/study degree\\ 
    die Lehrveranstaltung   & describes any teaching (e.g. lecture, seminar, etc.)\\
    die Vorlesung           & lecture\\
    das Seminar             & seminar\\
    das Kolloquium          & colloquim\\
    der Kurs                & course\\
    das Tutorium            & tutorial\\
    das Teilzeitstudium     & part-time studies\\
    die Hochschule          & not a university (one level lower)\\
    der Nachweis            & proof of (e.g. language skills)\\
    der Beweis              & proof (e.g. mathematical)\\
    die Geb\"uhr            & fee\\
\end{tabular}
\caption{\label{table:uni} university vocabulary}
\end{table}

\myspace{}
Phrases for writing:
\begin{itemize}
    \item[] Beschreibung der Grafik (description of a diagram):
    \item Die Grafik zeigt\dots
    \item Auf der Grafik sieht man\dots
    \item kontinuerlich steigen 
    \item sprunghaft gestiegen
    \item stark gesunken 

    \item[] General phrases:
    \item Abschlie{\ss}end/Zusammenfassend\dots
    \item Meiner Meinung nach\dots
    \item Nach meiner Meinung\dots
    
    \item[] Use different ways to express numbers:
    \item Der Umsatz von McDonalds hat sich von 2002 bis 2005 \textbf{halbiert}. 
    \item Von 1980 bis 1985 stieg der Umsatz von McDonalds \textbf{um das zwangzigfache}. 
    \item Von 1990 bis 1991 stieg der Umsatz von McDonalds \textbf{von} 1,0 \textbf{auf} 22,8 Milliarden USD. 
    \item use Konjuktiv I/II: Das wird unterschiedlich beurteilt.
    \item Der Anteil der Studierenden, deren Eltern gut verdienen, ist in den letzten 20 Jahren um ca. 20\% gestiegen.
\end{itemize}

\myspace{}
Useful resources:
\begin{itemize}
    \item FAU homepage: \url{https://www.fau.de/studium/}. Or any univeristy you want to apply to.
        Read it through, it has just the right vocabulary for the exam plus you might actually learn about the admissions :)
    \item official TestDaF page: \url{https://www.testdaf.de/zielgruppen/fuer-teilnehmende/vorbereitung/}.
        It already has original mock exams and a lot of useful tips. Perfect for practice and vocabulary.
        It also has example essay questions and answers. 
    \item More mock exams: \url{http://www.godaf.de/?url=/de/probe/}.
    
\end{itemize}

\end{document}